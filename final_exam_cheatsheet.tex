\documentclass[8pt]{report}
\usepackage{multicol}
\usepackage[a4paper, portrait, margin=0.5in]{geometry}
\setlength\parindent{0pt}
\setlength{\abovedisplayshortskip}{-1pt}
\setlength{\belowdisplayshortskip}{0pt} 
\usepackage{graphicx}
\usepackage{mathtools}
\pagenumbering{gobble}

\begin{document}
Criteria: Managers should manage their firm’s resources with the
ultimate objective of raising the firm’s market value

\begin{multicols}{2}
[
\section{Investment Decisions}
Time Value of Money - Time and Uncertainty (risk). Dollar today does not equal dollar tomorrow.
]
"Always Draw A Timeline To Visualize The Timing of Cashflows"
\begin{enumerate}
\item Identify timing and amount of the cash flows. Visualize them in a
timeline
\item Compute relevant discount factors for different points in time: $(1 + r)^{-t}$
\item Discount the single cash flows to time 0 using the corresponding
discount factors
\item Aggregate the cash flows at time 0
\end{enumerate}

$$c0----c1----c2----c3$$
$$|-----|-----|-----|$$


To move money forward: $$FV_t(c) = C \cdot (1+r)^t$$ 
To move money backward: $$PV_t(C) = \frac{C}{(1+r)^t}$$
Compounding: $$FV = \sum_t C_t \cdot (1+r)^{T-t}$$
Discounting: $$PV = \sum_t \frac{C_t}{(1+r)^{T-t}}$$

Perpetuity: $$PV_{perpetuity} = \frac{C}{r}$$
Important: The first cash flow arrives one period from today
Non-zero-start perpetuities: $$V_0 = \frac{(\frac{C}{r})}{(1+r)^t_{future}}$$ or $V_0 = C_{immediate} + \frac{C}{r}$

Annuity - a finite stream of cash flows of identical
magnitude and equal spacing in time: $$ PV_{annuity} = C \cdot \frac{1-(1+r)^{-T}}{r} $$
Growing perpetuity - when r $>$ g: $$ PV_{gp} = \frac{C}{r-g} $$
Growing annuity: $$ PV_{ga} = \frac{C}{r-g} \cdot (1 - (\frac{1+r}{1+g})^{-T}) $$ 
\end{multicols}
\hrule
\begin{multicols}{2}
[] 
Risky cash flows: \\
$ExpectedValue(CF) = p_1(C_1) + p_2(C_2)$

If you are risk averse, discount even higher rate: \\
$$\frac{C}{(1+r)+RP}$$ where RP is risk premium

In other words, risk needs compensation: \\
$PV = \sum_t \frac{E(C_t)}{(1+r+RP)^t}$

\end{multicols}

\begin{multicols}{2}
[]
Terminal value: steady state performance - FCF grow at constant rate

$NPV = PV(benefits - costs)$
$$
NPV = \sum_{t=0}^{T} E(FCF_t) / (1+k)^t
$$

$$
IRR = \sum_{t=0}^{T} E(FCF_t) / (1+IRR)^t == 0
$$

Accepting projects:
\begin{enumerate}
\item $NPV > 0$
\item $IRR > 0$ though IRR has problems with multiple roots
\item $PP < PredefinedCutoffPeriod$
\item $PI = \frac{NPV}{InitialCashOutlay} > 0$
\end{enumerate}
\end{multicols}

\begin{multicols}{2}
[
\section{Financing Decisions}
Bonds (debt) and Stocks (equity): remember  take the bootstrapping method for individual spot rates of a bond \\
Interest rate risk: interest rate and bond price are inversely related: no interest rate risk if you hold till maturity, and risk is lower for bonds with shorter maturities and larger coupons
]
The discount rate that equates the present value of a zero-coupon bond
with maturity T to its current market price P is called the spot rate.
$$
	sr_T = (\frac{F}{P_t})^{-1/T} - 1
$$

For coupon-bearing bond with varying annual rates:
$$
	P_T = \frac{CP}{1+y} + \frac{CP_2}{(1+y)^2} + ... + \frac{CP+F}{(1+y)^T}	
$$ where y is Yield-to-Market - the unique rate that makes PV of all future payments equal to price

Invested capital: $$ IC = PPE + WC $$ and \\
$ ROIC = \frac{NOPLAT}{IC} $ and \\
$ EBIT = OPREV - OPEXP - DEPR $

$$ NOPLAT = EBIT \cdot (1 - \tau) $$ where

 \begin{align*}
 NWC = Operating Cash+Inventory + \\ 
 Receivables + 
 Prepaid Expences  - \\ 
 Payables - Deferred Revenue - Accrued Expenses 
 \end{align*}
 

$$
	FCF = NOPLAT + DEPR - CAPEX - \Delta NWC
$$

Credit risk is $$ c_{debt} = risk_{free-rate} + credit spread $$
\end{multicols}
\hrule 
\begin{multicols}{2}
\paragraph{Stocks}
Returns: $ R = \frac{divididend + P_{t+1} - P_t} {P_t}$ \\
Covariance: $ Cov[R_A, R_B] = \sigma_{AB} == \rho_{AB}\sigma_A\sigma_B $ \\
Correlation: $ \rho_{AB} = \frac{\sigma_{AB}} {\sigma_A \sigma_B} $\\


The portfolio of two stocks, A and B: 
$$
	\mathbf{E}(R_p) = w_aE[R_a] + w_bE[R_b]
$$
$$
	\mathbf{V}(R_p) = w_a^2\sigma_a^2 + w_b^2\sigma_b^2 + 2 w_a w_b \sigma_{ab}
$$
Risk plane: $\{x=VOL, y=RATE\}$
\end{multicols}

\begin{multicols}{2}
\paragraph{CAPM}
Two-Fund Separation Theorem: the optimal portfolio needs only two funds: the risk-free asset and the Tangent Portfolio \\

\begin{align*}
Sharpe: SR_p = \frac{\mathbf{E}[R_p] - R_f}{\sigma[R_p]} \\
CAPM | SML: \mathbf{E}[R_i] = R_f + \beta_f \cdot (\mathbf{E}[R_m] - R_f)
\end{align*}

For CML vs SML: both explain expected returns and feature the risk-free rate viz the market portfolio
Differences: volatility vs beta, sharpe vs market premium, and efficient Portfolios vs all Stocks 
\end{multicols}


\begin{multicols}{2}
[
\section{Valuations}
]
Capital Structure \\
No taxes $(\tau=0)$

\begin{enumerate}
	\item MM1: V is independent of cap structure $D+E_L = V_L$ 
	\item MM2: $R_{E,L} = R_A + (R_A-R_D) \frac{D}{E_L}, where R_A=R_{E,U}$ 
	\item MM3: $WACC_L = \frac{E_L}{D+E_L} R_{E,L} + \frac{D}{D+E_L}R_D $
\end{enumerate}

With taxes $(\tau > 0)$ \\
\begin{enumerate}
	\item MM1: $ V+PV(ITS) = V_L$ 
	\item MM2: $R_{E,L} = R_A + (R_A-R_D) (1-\tau) \frac{D}{E_L} $ 
	\item MM3: $WACC_L = \frac{E_L}{D+E_L} R_{E,L} + \frac{D}{D+E_L}R_D(1-\tau) $	
\end{enumerate}

\end{multicols}
\hrule

\begin{multicols*}{2}
DCF, APV, and Relative Valuation using Multiples 
\begin{align*}
	& DCF = EV_0 = \frac{FCF_1}{(1+WACC)} + ... + \frac{FCF_T \cdot TV_T}{(1+WACC)^T} \\
	& APV: V_L = V_U + PV(ITS) \\
	& PV(ITS) = \frac{ITS_1}{(1+R_D} + ... + \frac{ITS_T + TV_T}{(1+R_D)^T} \\
\end{align*} 
where 
\begin{align*}
& WACC = \frac{E}{D+E} R_E + \frac{D}{D+E} \cdot (1-\tau) R_D \\
& TV_T = \frac{FCF_T \cdot (1+g)}{WACC-g} \\
& B_L = B_U \cdot (1+\frac{D}{E}), with \;  \beta_U = \beta_A
\end{align*}

For comparison:
\begin{enumerate}
\item Unlever the $\beta_L$ of comparable firms to get their $\beta_U$
\item Compute the average of $\beta_U$
\item Relever the $\beta_U$ with the firm's $\frac{D}{E}$ ratio
\end{enumerate}

Using Multiples:
\begin{enumerate}
\item multiple consistently defined
\item multiple uniformly estimated 
\item problems: difficult to define comparable firms, challenge quantifying differences in different multiples, backward-looking, you may overpay due to lack of fundamentals! 
\item example: $V_n = AvgComparable[\frac{EV}{EBIT}] \cdot EBIT_n$
\end{enumerate}
\end{multicols*}

\end{document}