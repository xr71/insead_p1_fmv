\documentclass[9pt]{article}
\usepackage{multicol}
\usepackage[a4paper, portrait, margin=0.5in]{geometry}
\setlength\parindent{0pt}
\setlength{\abovedisplayshortskip}{-1pt}
\setlength{\belowdisplayshortskip}{0pt} 
\usepackage{graphicx}

\begin{document}
Criteria: Managers should manage their firm’s resources with the
ultimate objective of raising the firm’s market value.

\begin{multicols}{2}
[
\section{Investment Decisions}
Time Value of Money - Time and Uncertainty (risk). Dollar today does not equal dollar tomorrow.
]
"Always Draw A Timeline To Visualize The Timing of Cashflows"
\begin{enumerate}
\item Identify timing and amount of the cash flows. Visualize them in a
timeline
\item Compute relevant discount factors for different points in time: $(1 + r)^{-t}$
\item Discount the single cash flows to time 0 using the corresponding
discount factors
\item Aggregate the cash flows at time 0
\end{enumerate}

$$c0----c1----c2----c3$$
$$|-----|-----|-----|$$


To move money forward: $$FV_t(c) = C \cdot (1+r)^t$$ 
To move money backward: $$PV_t(C) = \frac{C}{(1+r)^t}$$
Compounding: $$FV = \sum_t C_t \cdot (1+r)^{T-t}$$
Discounting: $$PV = \sum_t \frac{C_t}{(1+r)^{T-t}}$$

Perpetuity: $$PV_{perpetuity} = \frac{C}{r}$$
Important: The first cash flow arrives one period from today
Non-zero-start perpetuities: $$V_0 = \frac{(\frac{C}{r})}{(1+r)^t_{future}}$$ or $V_0 = C_{immediate} + \frac{C}{r}$

Annuity - a finite stream of cash flows of identical
magnitude and equal spacing in time: $$ PV_{annuity} = C \cdot \frac{1-(1+r)^{-T}}{r} $$
Growing perpetuity - when r > g: $$ PV_{gp} = \frac{C}{r-g} $$
Growing annuity: $$ PV_{ga} = \frac{C}{r-g} \cdot (1 - (\frac{1+r}{1+g})^{-T}) $$
\end{multicols}

\begin{multicols}{2}
[]
Risky cash flows: \\
$ExpectedValue(CF) = p_1(C_1) + p_2(C_2)$

If you are risk averse, discount even higher rate: \\
$\frac{C}{(1+r)+RP}$ where RP is risk premium

In other words, risk needs compensation: \\
$PV = \sum_t \frac{E(C_t)}{(1+r+RP)^t}$

\end{multicols}

\begin{multicols}{2}
[]
Terminal value: steady state performance - FCF grow at constant rate

$NPV = PV(benefits - costs)$
$$
NPV = \sum_{t=0}^{T} E(FCF_t) / (1+k)^t
$$

$$
IRR = \sum_{t=0}^{T} E(FCF_t) / (1+IRR)^t == 0
$$

\end{multicols}

\begin{multicols}{2}
[
\section{Financing Decisions}
Bonds (debt financing) and Stocks (equity financing)
]
The discount rate that equates the present value of a zero-coupon bond
with maturity T to its current market price P is called the spot rate.
$$
	sr_T = (\frac{F}{P_t})^{-1/T} - 1
$$

For coupon-bearing bond with varying annual rates:
$$
	P_T = \frac{CP}{1+y} + \frac{CP_2}{(1+y)^2} + ... + \frac{CP+F}{(1+y)^T}	
$$ where y is Yield-to-Market - the unique rate that makes PV of all future payments equal to price

Invested capital: $ IC = PPE + WC $ and \\
$ ROIC = \frac{NOPLAT}{IC} $ and \\
EBIT is $ EBIT = OPREV - OPEXP - DEPR $ and \\
NOPLAT is $ NOPLAT = EBIT \cdot (1 - \tau) $ and \\
Net WC is $ NWC = Operating Cash+Inventory+Receivables+Prepaid Expences - Payables - Deferred Revenue - Accrued Expenses $

$$
	FCF = NOPLAT + DEPR - CAPEX - \Delta NWC
$$

Credit risk is $$ c_{debt} = risk_{free-rate} + credit spread $$
\end{multicols}

\begin{multicols}{2}
\paragraph{Stocks}
Returns: $ R = \frac{divididend + P_{t+1} - P_t} {P_t}$ \\
Covariance: $ Cov[R_A, R_B] = \sigma_{AB} == \rho_{AB}\sigma_A\sigma_B $ \\
Correlation: $ \rho_{AB} = \frac{\sigma_{AB}} {\sigma_A \sigma_B} $\\


The portfolio of two stocks, A and B: 
$$
	\mathbf{E}(R_p) = w_aE[R_a] + w_bE[R_b]
$$
$$
	\mathbf{V}(R_p) = w_a^2\sigma_a^2 + w_b^2\sigma_b^2 + 2 w_a w_b \sigma_{ab}
$$
Risk plane: $\{x=VOL, y=RATE\}$
\end{multicols}

\begin{multicols}{2}
\paragraph{CAPM}
Two-Fund Separation Theorem: the optimal portfolio needs only two funds: the risk-free asset and the Tangent Portfolio


\end{multicols}


\begin{multicols}{2}
[
\section{Valuations}
TODO
]
TODO
\end{multicols}

\end{document}